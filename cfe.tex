\documentclass[twocolumn]{aastex61}
\input{preface}
\begin{document}
Sgr B2 Cluster Formation Efficiency

\begin{abstract}
    The fraction of stars forming in dense, gravitationally bound clusters is
    an important parameter in understanding both the star formation history of
    the universe and the effects of stellar feedback from groups of stars.
    (maybe something about the K+ theory)
    We report a measurement of the cluster formation efficiency (CFE), the
    fraction of stars forming in clusters, in the highest-density region in
    the Galaxy, Sgr B2.  We find that about a third of the stars (37-43\%) in
    Sgr B2 are forming in bound clusters, a value consistent with the
    predictions of the \citet{Kruijssen2012a} models.
\end{abstract}

\section{Introduction}

{\color{red}TODO: fill out introduction}
The cluster formation efficiency is an important measurement...
Kruijssen+ 2012

{\color{red}
Notes to self:
\begin{itemize}
    %\item mass of stars, gas in M, N, distributed population, S? [mostly taken from Schmiedeke] [DONE?]
    %\item virial parameters?  Use DePree's data for velocity dispersion?  [added my own H41a; DePree's are useful though] [ DONE - note to Diederik - this is for stars, not gas ]
    %\item cluster definition: purely radial.  Is S a cluster?  Is NE?  [assuming "No" below; have added discussion about effect of expanding N]
    \item Age difference.  What (maximal) errors can this impose?
\end{itemize}
}

\section{Observational Summary}

We use the catalogs described in \citet{Ginsburg2018a}, \citet{Gaume1995a}, and
\citet{De-Pree2015a} to infer the total stellar population.

\citet{Gaume1995a} observed Sgr B2 at 1.3 cm with $\sim0.25$\arcsec resolution
with the VLA.  They detected 49 continuum sources.  These are exclusively \hii
regions and components of \hii regions.  \citet{De-Pree2015a} used 7 mm JVLA
Q-band observations at 0.05\arcsec resolution to catalog 26 sources in Sgr B2 M
and 5 in Sgr B2 N.  Of these, 7 detected in Sgr B2M were not reported in
\citet{Gaume1995a} because they were not resolved.  We assume each of the
VLA-detected sources is an \hii region and therefore contains at least one star
with $M\gtrsim20$ \msun, equivalent to a B0 star.

\citet{Ginsburg2018a} reported a total of 271 sources spread throughout the Sgr
B2 cloud, of which 31 are confirmed \hii regions.  The rest are candidate
protostars, with luminosity constraints and other considerations suggesting
that they all have $M\gtrsim8$ \msun.


\section{The mass of the clusters}
Cluster masses are determined by counting the number of HII regions and high-mass
protostellar cores associated with each of the clusters.

In Table 2 of \citet{Ginsburg2018a}, four clusters were considered: N, M, NE, and
S.  Here, we re-evaluate the ``clusters" in NE and S.  These regions are not
centrally concentrated and do not have many sources within the named region.
They are both moderate mass and, at present, do not appear likely to form bound
clusters.  We therefore exclude them from the analysis {\color{red} or, do with/without
them}.

\begin{table*}[htp]
\centering
\caption{Cluster Masses}
\begin{tabular}{cccccc}
\label{tab:clustermassestimates}
Name & $N(cores)$ & $N(H\textsc{ii})$ & $M_{inferred, cores}$ & $M_{inferred, H\textsc{ii}}$ & $M_{inferred,max}$ \\
 &  &  & $\mathrm{M_{\odot}}$ & $\mathrm{M_{\odot}}$ & $\mathrm{M_{\odot}}$ \\
\hline
M & 17 & 47 & 2300 & 15000 & 15000 \\
N & 11 & 3 & 1500 & 980 & 1500 \\
NE & 4 & 0 & 540 & 0 & 540 \\
S & 5 & 1 & 680 & 330 & 680 \\
Unassociated & 203 & 6 & 27000 & 2000 & 27000 \\
Total & 240 & 57 & 33000 & 19000 & 46000 \\
Clustered with NE, S & 37 & 51 & 5000 & 17000 & 18000 \\
Clustered only M, N & 28 & 50 & 3800 & 16000 & 17000 \\
\hline
\end{tabular}
\par
Partial reproduction of Table 2 in \citet{Ginsburg2018a}. $M_{inferred,cores}$ and $M_{inferred,\hii}$ are the inferred total stellar masses assuming the counted objects represent fractions of the total mass 0.09 (cores) and 0.14 (\hii regions).$M_{inferred,max}$ is the greater of these two.    The \emph{Total} row represents the total over the whole observed region.    The two Clustered rows show the total inferred mass of clusters including all four candidate clusters including NE and S, then the mass of clusters including only Sgr B2 M and N.
\end{table*}


\subsection{Cluster membership in Sgr B2 M}
\citet{Schmiedeke2016a} marked the Sgr B2 M cluster as a 13\arcsec  (0.5 pc) radius
region centered on Sgr B2 M F3.  Within this volume, there are 47 \hii regions
in the joint  \hii region catalogs \citep{Gaume1995a,De-Pree2015a} from their
0.05\arcsec resolution 7 mm Q-band
VLA observations.  There are 17 non-\hii-region cores, the faintest of which is
1.3 mJy at 3 mm \citep{Ginsburg2018a}.  By extrapolating the \hii region counts,
\citet{Ginsburg2018a} inferred a total stellar mass of 1.5\ee{4} \msun.
%somewhat lower than the 2\ee{4} \msun reported by \citet{Schmiedeke2016a};
%we adopt the .

% Column density calculation
% 40" = 1.6pc
% ((2.3e24*u.cm**-2) * (40*u.arcsec/2.35 * 8.5*u.kpc)**2 * 2*np.pi * 2.8*u.Da).to(u.Msun, u.dimensionless_angles())
% {\color{red} There is some inconsistency in \citet{Schmiedeke2016a} in these
% next two paragraphs.}
% 
% Based on \citet{Schmiedeke2016a}'s column density measurement of
% $N(\hh)=2.3\ee{24}~\persc$
% in a 40\arcsec beam toward Sgr B2 M,
% % this is from Table 3, using the 3D model
% the total gas mass in the center-most beam is $M_{gas, M} = 1.6\ee{5} \msun$.
% The instantaneous SFE is only $M_*/M_{gas}=1\%$.

% Table B.3. gives: 
% M2: ((2e6*u.cm**-3 * 2.8 * u.Da) * 4/3*(2e4*u.au)**3).to(u.M_sun)
% <Quantity 167.00206886275507 solMass>
% M1: ((2e6*u.cm**-3 * 2.8 * u.Da) * 4/3*(3e4*u.au)**3).to(u.M_sun)
% <Quantity 563.6319824117984 solMass>

% Table 2 gives:
\citet{Schmiedeke2016a} give a gas mass of $M=9.6\ee{3}$ \msun in the Sgr B2 M
cluster in their Table 2 and measure an instantaneous SFE of about 60\%, assuming
a cluster radius $r=0.5$ pc.  The total mass within 0.5 pc is then about $M_M =
2.5\ee{4}$ \msun, and the escape speed is $v_{esc}=14~\kms$.


% While it is possible that some of these sources are unassociated with Sgr B2,
% their proximity to the Sgr B2 core suggests they are indeed bound...

% mass measurement:
% (7e3*u.M_sun/u.pc**3 * (1.4*u.pc)**3 * 4/3.*np.pi)
It is possible the Sgr B2 M cluster is substantially larger, 35\arcsec (1.4 pc).
Within this radius, the `core' count is larger, 52 rather than 17, but the \hii
region count increases only marginally, from 47 to 49.  By contrast,
the gas mass is larger, $M_{gas,1.4pc} = 8\ee{4}~\msun$, so the integrated SFE is lower,
about 20\%.  
%Since the \hii region-inferred
%stellar mass is larger in both cases, we use this estimate, but
The presence of many cores in the outskirts of the Sgr B2 M cluster suggests
both that it may grow in stellar mass by accretion by up to an additional
$\sim50\%$ and that the lack of cores in the innermost region is due to
incompleteness (e.g., from confusion) rather than their absence, as suggested
in \citet{Ginsburg2018a}.

\subsection{Cluster membership in Sgr B2 N}
\citet{Schmiedeke2016a} marked the Sgr B2 N cluster as a 10\arcsec  (0.4 pc) radius circle
centered on Sgr B2 N K2.  \citet{Schmiedeke2016a} identified 3 compact \hii regions
and \citet{Ginsburg2018a} identified 11 cores within this region.  The inferred
total stellar plus protostellar mass is 980-1500 \msun.  However, unlike Sgr B2
M, Sgr B2 N is gas-dominated, with $M_{gas,N} = 2.8\ee{4}~\msun$ and SFE
$\sim5\%$ \citep{Schmiedeke2016a}.  The escape speed from the 0.4 pc cluster is
$v_{esc} = 18~\kms$.

Sgr B2 N is therefore better described as a `protocluster', in contrast with
Sgr B2 M, which is a (very) young massive cluster (YMC).  Sgr B2 N will need to
form an additional several thousand \msun of stars to form a YMC, and will need
to do so at high efficiency.  However, since there is evidence that the
protocluster itself is still rapidly accreting both stars and gas, this outcome
is quite likely.


\subsection{Velocity Dispersion Measurements - boundedness}

We compare our velocity measurements to those of \citet{De-Pree2011a} and
\citet{De-Pree1996a} and perform some new velocity measurements based on the
2013.1.00269.S data from
\citet{Ginsburg2018a}.  Of the 32 unique regions within the field
identified in the \citet{Gaume1995a} 1.3 cm data, which have resolution comparable
to the \citet{De-Pree2011a} H52$\alpha$ and H66$\alpha$ and the \citet{Ginsburg2018a}
H41$\alpha$ data, 15 had measurements in \citet{De-Pree2011a}.  We have
measured an additional 11 RRL velocities from the H41$\alpha$ line.
Our measurements agree to within 5 \kms with those of \citet{De-Pree2011a} for
all sources
we both measured except F10.37, for which we measure a $\sim20~\kms$ discrepancy;
our spectrum is of much higher quality, so we adopt the H41$\alpha$ measurement
as correct.
All measured velocities are reported in Table \ref{tab:h41afits}.

We measure the 1D velocity dispersion in Sgr B2 by taking the standard deviation
of the measured V$_{LSR}$ values.
Using only the \citet{De-Pree2011a} measurements, we obtain $\sigma_{1D}\approx9~\kms$.
Using the full data set, we obtain a higher $\sigma_{1D}\approx12~\kms$.
In both cases, $\sigma_{1D}$ is significantly lower than the escape velocity.
%1D Velocity Dispersion of 41a: 11.773405291227526, 52a: 9.013305133051228, 66a: 9.52127673916695

However, some individual sources are moving at high velocity with respect to
the average ($\bar{v}_{LSR}(H41\alpha) = 58.5 \kms$, $\bar{v}_{LSR}(H52\alpha)
= 65.8 \kms$), the fastest being G10.47 at $v_{LSR}=34$ \kms or
$v_{center}=24-32$ \kms.  There is a small group at these highly negative
velocities and a projected distance from the center $r<0.1$ pc; these may be
bound to a higher potential than we have inferred above, or they could be unbound
from the main cluster.
The \hii region J is separated by 0.4 pc and 16--24 \kms and is a diffuse \hii
region.  It may not be connected with the rest of the cluster.
If we exclude regions J, F10.37, G, G10.44, and G10.47, the velocity dispersion
drops to $\sigma_{1D}\approx8~\kms$.

\clearpage
\begin{table}[htp]
\caption{H41$\alpha$ Line Fits}
\begin{minipage}{130mm}
\begin{tabular}{llllllllllllllllll}
\label{tab:h41afits}
Source & Coordinates & $v_\mathrm{LSR}$(41) & $\sigma\left[v_\mathrm{LSR}(41)\right]$ & $\mathrm{FWHM}$(41) & $\sigma\left[\mathrm{FWHM}(41)\right]$ \\
 &  & $\mathrm{km\,s^{-1}}$ & $\mathrm{km\,s^{-1}}$ & $\mathrm{km\,s^{-1}}$ & $\mathrm{km\,s^{-1}}$ \\
\hline
A1 & 17:47:19.436 -28:23:01.36 & 62.7 & 0.5 & 31.6 & 1.1 \\
A2 & 17:47:19.566 -28:22:55.95 & 58.1 & 0.7 & 26 & 1.5 \\
B & 17:47:19.907 -28:23:02.91 & 75.6 & 0.4 & 34.9 & 0.9 \\
B10.06 & 17:47:19.868 -28:23:01.41 & 49.7 & 1.6 & 30.5 & 3.7 \\
B10.10 & 17:47:19.908 -28:23:02.13 & 70.1 & 1.7 & 27.3 & 3.9 \\
B9.96 & 17:47:19.776 -28:23:10.18 & 58.3 & 1.2 & 29.6 & 2.8 \\
B9.99 & 17:47:19.802 -28:23:06.9 & 61.7 & 0.8 & 23.2 & 1.8 \\
D & 17:47:20.053 -28:23:12.87 & 64.3 & 0.7 & 33.7 & 1.5 \\
E & 17:47:20.071 -28:23:08.65 & 61.3 & 0.3 & 29.3 & 0.8 \\
F1 & 17:47:20.12 -28:23:04.26 & 80.6 & 1.2 & 72.4 & 2.8 \\
F10.303 & 17:47:20.112 -28:23:03.7 & 57.2 & 1.1 & 81 & 2.7 \\
F10.33 & 17:47:20.14 -28:23:06.1 & 55.2 & 2.3 & 36.4 & 5.4 \\
F10.35 & 17:47:20.156 -28:23:06.73 & 58.7 & 7.1 & 72.2 & 17 \\
F10.37 & 17:47:20.179 -28:23:05.95 & 39.8 & 1 & 58 & 2.4 \\
F10.39 & 17:47:20.195 -28:23:06.65 & 63.1 & 1.4 & 53.5 & 3.3 \\
F2 & 17:47:20.17 -28:23:03.75 & 78.2 & 1.8 & 98.4 & 4.2 \\
F3 & 17:47:20.176 -28:23:04.81 & 61.1 & 0.4 & 45.2 & 1 \\
F4 & 17:47:20.219 -28:23:04.34 & 66.9 & 0.4 & 42.6 & 0.9 \\
G & 17:47:20.287 -28:23:03.07 & 44.3 & 0.5 & 43.8 & 1.3 \\
G10.44 & 17:47:20.246 -28:23:03.36 & 39.6 & 1.2 & 32.6 & 2.9 \\
G10.47 & 17:47:20.274 -28:23:02.38 & 34.1 & 1.6 & 22.4 & 3.7 \\
I & 17:47:20.511 -28:23:06.08 & 60.2 & 0.3 & 31.1 & 0.6 \\
I10.52 & 17:47:20.329 -28:23:08.14 & 60.9 & 2.1 & 22.8 & 4.9 \\
J & 17:47:20.574 -28:22:56.17 & 42.5 & 2 & 28.4 & 4.7 \\
\hline
\end{tabular}
\\
Velocities measured from radio recombination line fits for the HII regions in Sgr B2.  The H41$\alpha$ line comes from the ALMA data of \citet{Ginsburg2018a}.
\end{minipage}
\end{table}

\clearpage

\subsection{The cluster formation efficiency}
Table \ref{tab:clustermassestimates} shows the breakdown of ongoing star
formation within the Sgr B2 region.  The total inferred mass of recently
formed or forming stars is $M_{*,total}\approx4.6\ee{4}$ \msun spread across
the whole cloud, with $M_{*,clustered}\approx1.7\ee{4}$ \msun concentrated
in the Sgr B2 M and N clusters.  These values imply a \textit{cluster
formation efficiency} $CFE=M_{*,clustered}/M_{*,total} = 37\%$.

We have noted above that the membership of clusters M and N could be expanded,
and while this expansion would have no effect on the estimated mass of the clusters
(because their masses have been inferred from more complete samples of \hii regions),
it would reduce the number of unassociated cores by about 25\%, increasing the inferred
CFE to $\approx43\%$. % this number calculated by hand: m_tot = 46k - (27k*0.25) = 40, 17 / 40 = 43%.
Treating Sgr B2 NE and S as clusters would also serve to increase our CFE estimates,
but only by 2-3\%, so we ignore them.

\subsection{Comparison of observations to predictions}
{\color{red} This is the section where we'll pull in the predictions from 
\citet{Kruijssen2012a} and compare them to the observations.}

\input{solobib.tex}

\end{document}
