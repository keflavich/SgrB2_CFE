\begin{table*}[htp]
\centering
\caption{Cluster Masses}
\begin{tabular}{cccccc}
\label{tab:clustermassestimates}
Name & $N(cores)$ & $N(H\textsc{ii})$ & $M_{inferred, cores}$ & $M_{inferred, H\textsc{ii}}$ & $M_{inferred,max}$ \\
 &  &  & $\mathrm{M_{\odot}}$ & $\mathrm{M_{\odot}}$ & $\mathrm{M_{\odot}}$ \\
\hline
M & 17 & 47 & 2300 & 15000 & 15000 \\
N & 11 & 3 & 1500 & 980 & 1500 \\
NE & 4 & 0 & 540 & 0 & 540 \\
S & 5 & 1 & 680 & 330 & 680 \\
Unassociated & 203 & 6 & 27000 & 2000 & 27000 \\
Total & 240 & 57 & 33000 & 19000 & 46000 \\
Clustered with NE, S & 37 & 51 & 5000 & 17000 & 18000 \\
Clustered only M, N & 28 & 50 & 3800 & 16000 & 17000 \\
\hline
\end{tabular}
\par
Partial reproduction of Table 2 in \citet{Ginsburg2018a}. $M_{inferred,cores}$ and $M_{inferred,\hii}$ are the inferred total stellar masses assuming the counted objects represent fractions of the total mass 0.09 (cores) and 0.14 (\hii regions).$M_{inferred,max}$ is the greater of these two.    The \emph{Total} row represents the total over the whole observed region.    The two Clustered rows show the total inferred mass of clusters including all four candidate clusters including NE and S, then the mass of clusters including only Sgr B2 M and N.
\end{table*}
